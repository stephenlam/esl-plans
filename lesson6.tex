\documentclass{tufte-handout}
\usepackage{expex}
\usepackage{hyperref}
\usepackage{framed}
\usepackage{booktabs}

\title{Lesson 6: Countable/uncountable objects, ``How much/many'', simple present tense\thanks{Material in this lesson is adapted from Kenneth Beare's Absolute Beginners 20 Point Program on About.com. This lesson guide covers material from lessons 17, 18, and 20. Page numbers refer to \emph{English in Action}, 2nd ed. by Foley and Neblett.}}
\author{Stephen Lam}
\date{16 June 2014}

\newcommand{\dialt}{\smallcaps{teacher}\hspace{1em}}
\newcommand{\dials}{\smallcaps{student}\hspace{1em}}
\newcommand{\dialsone}{\smallcaps{s1}\hspace{1em}}
\newcommand{\dialstwo}{\smallcaps{s2}\hspace{1em}}

\begin{document}

\maketitle

\begin{abstract}
\noindent By the end of this lesson, students will be able to distinguish between countable and uncountable nouns, and should be able to ask and answer questions beginning with \emph{how many} and \emph{how much}. They will be introduced to present simple tense.
\end{abstract}

\section{Some/any (24 min total)}
On the board, write two headers: ``some'' and ``4''. Put a list of uncountable words under ``some'' and a list of countable words under ``4''.

\begin{margintable}
  \fontfamily{ppl}\selectfont
  \begin{tabular}{ll}
    \toprule
    some & 4 \\
    \midrule
    water & apples \\
    wine & oranges \\
    cheese & gloves \\
    bread & beans \\
   	grass & shoes \\
    money & coins \\
    corn & books\\
    rice\\
    \bottomrule
  \end{tabular}
\end{margintable}

\subsection{Countable objects (8 min)}
\begin{framed}
\dialt Are there \textbf{any} oranges in this picture?\\
\hspace{0em}\hphantom \dialt Yes, there are \textbf{some} oranges in that picture.\\
\dialt Are there \textbf{any} apples in this picture?\\
\hspace{0em}\hphantom \dialt No, there aren't \textbf{any} apples in that picture.\\
\end{framed}

\subsection{Uncountable objects (8 min)}
Point out the list on the board at this time.
\begin{framed}
\dialt Is there \textbf{any} water in this picture?\\
\hspace{0em}\hphantom \dialt Yes, there is \textbf{some} water in that picture.\\
\dialt Is there \textbf{any} cheese in this picture?\\
\hspace{0em}\hphantom \dialt No, there isn't \textbf{any} cheese in that picture.\\
\end{framed}

\subsection{Students ask questions (8 min)}

\section{``How many'' and ``how much'' (24 min total)}
\subsection{Countable objects (8 min)}
\begin{framed}
\dialt \textbf{How many} apples are in this picture?\\
\hspace{0em}\hphantom \dialt \textbf There are \textbf{4} apples in that picture.\\
\hspace{0em}\hphantom \dialt \textbf There are \textbf{a few} apples in that picture.\\
\hspace{0em}\hphantom \dialt \textbf There are \textbf{lots of} apples in that picture.\\
\hspace{0em}\hphantom \dialt \textbf There are \textbf{no} apples in that picture.\\
\end{framed}

\subsection{Uncountable objects (8 min)}
\begin{framed}
\dialt \textbf{How much} wine is in the glass?\\
\hspace{0em}\hphantom \dialt \textbf There is \textbf{a little} wine in the glass.\\
\hspace{0em}\hphantom \dialt \textbf There is \textbf{a lot of} wine in the glass.\\
\hspace{0em}\hphantom \dialt \textbf There is \textbf{no} wine in the glass.\\
\end{framed}

\subsection{Combining into a mini-conversation (8 min)}
\begin{framed}
\dialt Is there any water in this picture?\\
\hspace{0em}\hphantom \dialt Yes, there is some water in that picture.\\
\dialt How much water is in this picture?\\
\hspace{0em}\hphantom \dialt There is a lot of water in that picture.
\end{framed}

\section{Stretch Break! (2 min total)}

\section{Simple Present tense (30 min total)}
\subsection{I, you, they}
\begin{framed}
\dialt What do you do?\\
\hspace{0em}\hphantom \dialt I \textbf{teach} English.\\
\dialt What do they do?\\
\hspace{0em}\hphantom \dialt They \textbf{study} English.\\
\end{framed}

\ex
Ask individual students:\\
\dialt Lucia, what do you do?\\
\dials I study English.\\
\dialt Lucia, what do I do?\\
\dials You teach English.
\xe

Use other verbs, showing the corresponding illustration (van driver, taxi driver, etc.).

\subsection{He, she}
Emphasize the -s verb conjugation.
\begin{framed}
\dialt What does he do? (taxi driver)\\
\hspace{0em}\hphantom \dialt He drive\textbf{s} a taxi.\\
\dialt What does he do? (van driver)\\
\hspace{0em}\hphantom \dialt He drive\textbf{s} a van.\\
\dialt What do she do? (babysitter)\\
\hspace{0em}\hphantom \dialt She watch\textbf{es} children.\\
\dialt What does he do? (plumber)\\
\hspace{0em}\hphantom \dialt He repair\textbf{s} bathrooms.\\
\dialt What does he do? (cook)\\
\hspace{0em}\hphantom \dialt He prepare\textbf{s} food.\\
\dialt What does she do? (waitress)\\
\hspace{0em}\hphantom \dialt She serve\textbf{s} food.\\
\dialt What does he do? (busboy)\\
\hspace{0em}\hphantom \dialt He clean\textbf{s} and clear\textbf{s} tables.\\
\dialt What does she do? (housekeeper)\\
\hspace{0em}\hphantom \dialt She clean\textbf{s} and vacuum\textbf{s} rooms.\\
\dialt What does he do? (laundry worker)\\
\hspace{0em}\hphantom \dialt He wash\textbf{es} and dri\textbf{es} sheets and towels.\\
\end{framed}

\end{document}

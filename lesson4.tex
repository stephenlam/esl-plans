\chapter{Lesson 4: Jobs, ``Who'' and ``What''}

\begin{abstract}
\noindent By the end of this lesson, students will know basic jobs vocabulary. They will be able to ask questions about others' occupations using the interrogative pronous \emph{who} and \emph{what}.
\end{abstract}

\section{Jobs (20 min total)}
\subsection{Vocabulary (10 min)}
\marginnote[2.8em]{While pointing to the photos of various jobs on page 172.}
\begin{framed}
\dialt She is a desk clerk. She is a babysitter. He is a busboy\ldots
\end{framed}

Model the idea of ``repeating after me'', giving students a new class instruction that they will understand in the future.
\ex
\dialt \textbf{Repeat after me.} She is a desk clerk.\\
\dials She is a desk clerk.\\
\dialt She is a babysitter.\\
\dials She is a babysitter.\\
\emph{etc\ldots}
\xe

\subsection{Asking questions about jobs (10 min)}
Model then ask individual students:
\ex
\dialt Sarah, is he a cashier? (\emph{point to picture})\\
\dials Yes, he is a cashier.
\xe
\ex
\dialt Sarah, is she a babysitter? (\emph{point to picture})\\
\dials No, she isn't a babysitter. She is a waitress.
\xe
Students ask each other (give S1 a picture):
\pex
\dialt S1, ask S2 a question.
\a \dialsone Is she a waitress? (\emph{point to picture})\\
\dialstwo Yes, she is a waitress.
\a \dialsone Is she a waitress? (\emph{point to picture})\\
\dialstwo No, she isn't a waitress. She is a teacher.
\xe

\section{``Who'' and ``What'' questions (25 min total)}
\subsection{``What are you?'' (10 min)}
\begin{framed}
\dialt Stephen, \textbf{what} are you?\\
\hspace{0em}\hphantom \dialt I am a \textbf{teacher}.
\end{framed}

Ask individual students:\marginnote{Some students' jobs may not included in the vocabulary lesson. If this happens, point to a picture and then model a question pretending to be something from one of the pictures.}
\ex
\dialt What are you?\\
\dials I am a cashier.
\xe
Have students ask each other.

\subsection{``Who is a \ldots?'' (10 min)}
\begin{framed}
\dialt \textbf{What} are you? I am a \textbf{teacher}.\\
\hspace{0em}\hphantom \dialt \textbf{Who} is a waitress? \textbf{Dita} is a waitress.
\end{framed}
Do this exercise with each student:
\ex
\dialt Who is a cook?\\
\dials \underline{\hspace{1.5cm}} is a cook.\\
\dialt Who is a teacher?\\
\dials You are a teacher.
\xe

\subsection{Mixing up \emph{who} and \emph{what} (5 min)}
Ask students questions using \emph{who} and \emph{what}, alternating between the two.

\section{Stretch Break! (5 min total)}
Use this time also to mention body part names (``head'', ``neck'', ``arms'', ``fingers'') as each part is stretched. At the end, careful listeners will be rewarded with candy.

\pagebreak
\section{Greetings (30 min total)}
\subsection{Basic Greetings (15 min)}
These exercises will combine job vocabulary with new concepts. Use gestures (thumbs up) and facial expressions to help students understand the differences between the following:
\begin{framed}
\dialt Hello, how are you? Hi, I'm \textbf{fine}.
\end{framed}
\begin{framed}
\dialt Hi, how are you? Hello, I'm \textbf{OK}.
\end{framed}
\begin{framed}
\dialt Hi, how are you? Hi, I'm \textbf{well}.
\end{framed}

Teachers and students stand up and ask each other.

Now combine concepts into a longer conversation:
\begin{framed}
\dialt Hello Stephen, how are you?\\
\hspace{0em}\phantom \dialt Hello, I'm fine.\\
\hspace{0em}\phantom \dialt What is this?\\
\hspace{0em}\phantom \dialt That's a book.\\
\hspace{0em}\phantom \dialt What are you?\\
\hspace{0em}\phantom \dialt I'm a teacher.\\
\hspace{0em}\phantom \dialt Bye.\\
\hspace{0em}\phantom \dialt Bye.
\end{framed}

Ask individual students:
\ex
\dialt Hello Francisco, how are you?\\
\dials Hi, I'm fine.\\
\dialt What is this?\\
\dials That is a pencil.\\
\dialt What are you?\\
\dials I'm a van driver.\\
\dialt Bye, Francisco.\\
\dials Bye.
\xe

Now we introduce names of nations, cities, \emph{etc\ldots}:
\marginnote[3em]{Repeat with different country and city names}
\begin{framed}
\dialt \textbf{Where} are you \textbf{from}?\\
\hspace{0em}\phantom \dialt I am \textbf{from} the USA/Virginia/Chantilly.
\end{framed}

Students ask each other.

\subsection{Asking questions with \emph{he} and \emph{she} (5 min)}
\begin{framed}
\dialt Where is Paolo from?\\
\hspace{0em}\phantom \dialt \textbf{He} is from Italy.
\end{framed}
Ask individual students:
\ex
\dialt Dita, where is Cecilia from?\\
\dials She is from Brazil.\\
\dialt Where are you from?\\
\dials I'm from El Salvador.
\xe
Have students ask each other.

\subsection{Nationalities (10 min)}
\begin{framed}
\dialt Where are you from?\\
\hspace{0em}\phantom \dialt I am from the USA.\\
\hspace{0em}\phantom \dialt Are you \textbf{American}?\\
\hspace{0em}\phantom \dialt No, I am \textbf{Chinese}.\\
\end{framed}
\marginnote[2.2em]{Go through the various nationalities in the classroom. This is also an opportunity to repeat the negative forms of 'to be' for 'he' and 'she'. Also go through the list of nations and nationalities having students repeat after you to focus on pronunciation for a moment.}
\begin{framed}
\dialt Where is Francisco from?\\
\hspace{0em}\phantom \dialt He is from \textbf{El Salvador}.\\
\hspace{0em}\phantom \dialt Is he \textbf{American}?\\
\hspace{0em}\phantom \dialt No, he isn't \textbf{American}. He is \textbf{Salvadoran}.
\end{framed}

Ask individual students using positives first, then introduce and review the negatives. Then have students ask each other.

Expand the conversation:
\ex
\dialt Dita, hi, how are you?\\
\dials Hi, I'm fine.\\
\dialt Where are you from?\\
\dials I'm from Portugal.\\
\dialt Are you American?\\
\dials No, I'm not American. I'm Portuguese.\\
\dialt What are you?\\
\dials I'm a laundry worker.\\
\dialt What's this?\\
\dials That's a book.\\
\dialt Bye.\\
\dials Bye.
\xe

\section{Personal Information (15 min total)}
\begin{framed}
\dialt What is your \textbf{telephone number}?\\
\hspace{0em}\phantom \dialt My \textbf{telephone number} is 571-358-7232.
\end{framed}
\ex
\dialt Susan, hi, how are you?\\
\dials Hi, I'm fine.\\
\dialt What is your telephone number?\\
\dials My telephone number is \underline{\hspace{2cm}}.\\
\xe

\begin{framed}
\dialt What is your \textbf{address?}\\
\hspace{0em}\phantom \dialt My \textbf{address} is 14016 Eagle Chase Circle.
\end{framed}
\ex
\dialt Susan, hi, how are you?\\
\dials Hi, I'm fine.\\
\dialt What is your address?\\
\dials My address is \underline{\hspace{3cm}}.\\
\xe

\ex
\dialt Susan, hi, how are you?\\
\dials Hi, I'm fine.\\
\dialt What is your address?\\
\dials My address is 32 14th Avenue.\\
\dialt What is your telephone number?\\
\dials My telephone number is 587-8945.\\
\dialt Where are you from?\\
\dials I'm from Russia.\\
\dialt Are you American?\\
\dials No, I'm not American. I'm Russian.\\
\dialt What are you?\\
\dials I'm a nurse.\\
\dialt What's this?\\
\dials That's a book.\\
\dialt Bye.\\
\dials Bye.
\xe